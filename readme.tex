\documentclass[a4paper,12pt]{article}
 
% let's use some nice packages
\usepackage{times}
\usepackage{amssymb}
\usepackage{amsmath}
% set all margins to 2.5 cm
\usepackage[margin=2.5cm]{geometry}
% for fancy and easy modification of header and footer
\usepackage{fancyhdr}
% include hyperlinks in pdf file
\usepackage{hyperref}
% decide if run PdfTeX or LaTeX
\ifx\pdfoutput\undefined
% we are running LaTeX, not pdflatex
\usepackage{graphicx}
\else
% we are running pdflatex, so convert .eps files to .pdf
\usepackage[pdftex]{graphicx}
\usepackage{epstopdf}
\usepackage{listings}
\fi
 
% also include colors
\usepackage{color}
 
% define a path for your images
\graphicspath{{.},{../images/},,{../../images/}{../../anotherdir/}}
 
% define a new command
\newcommand{\mylab}[1]{\label{#1}}
 
\title{SSL+PHP Web Server}
\author{Steven Lu (sjlu@eden.rutgers.edu)}
\date{\today}
 
\begin{document}
% show the title, author and date
\maketitle
 
\section{Introduction}
 
The following assignment implements SSL and PHP into a deliverable 
web server. This was inspired by the lectures of IO, directories, Web servers,
threads and SSL. Another major inspiration was that modern web servers can
process dynamic data (or have scripts). In this case, I wanted to implement
PHP or Pre-hypertext Processor execution into the server.\\
\\
Since we implemented a very small web server that serves a single static file,
the web server itself had to be implemented first, being able to handle the 
header request from the browser, process it, and deliver acceptable content
to the client's browser. This also means that we have to also handle the
response headers properly (such as 404 Not Found, 200 OK and 500 Internal 
Server Error).\\
\\
The second major component was implementing SSL encryption into the server.
To do so, we had to import certain C libraries such as GNU TLS in order
to handle the encryption. This means that we had to detect the incoming
SSL request, create a handshake with the browser, encrypt the data based
on the session and then finally deliver the data through a secure pipe.\\
\\
The last major component had to deal with implementing safe script execution
into the web server. This means that we had to detect that the following
request from the client was calling a PHP script. If it was, we would
execute the script and take its output and send it to the client.\\
\\
Some minor components include threading the web server, allowing it
to handle multiple and simultaneous connections at time so that the
implementation did not seem slow to the client. GNU TLS functions in 
this case for encryption can sometimes be fairly slow, this is why
multi-threading was required.\\
\\
\section{Usage}
Very simplistic usage, to run the application.\\
\\
\begin{lstlisting}
./web-server
\end{lstlisting}

Put all your files in the content directory, the web server
will serve content only in that directory.\\
\\
Place your browser to \url{https://localhost:8443/index.html}

\section{Implementation}
\subsection{Web Server Implementation}

For clarification purposes, this section will talk about how we handle
the data and not the actual file descriptors themselves because due to
the SSL implementation, GNU TLS handles most of the client file descriptors
for us already. For example, a typical web call would call send() to the client,
but in this case we call gnutls-record-send() in order to properly send
over encrypted data to the pipe. The inputs are very similar except for the
file descriptor. In a normal case the file descriptor would be an integer
while in the SSL case, it is of a type gnutls-session-t.\\
\\
When all the established connections are completed, we take in the request
headers from the client browser using gnutls-record-recv (which is called
by the function readline). We should note that read-line takes in one
character at a time and appends it to a line buffer. Because of the time
limitations, we only take in the first line which contains the file request
header. This header looks something like "GET /index.html HTTP/1.1".\\
\\
After readline has finished giving us our request header, we need to
process it accordingly. We do this by splitting up the header by "spaces",
using a string tokenizer. The first part of the header is the request type.
Request types include GET, POST, PUT and DELETE. However, due to the 
nature of time, the only request type supported is GET. To implement
POST and other parameters, we must have to process the request of
header and properly organize it accordingly until we get to the
data parameter for POST. So for now, we are limited to GET.\\ 
\\
The second part of the header gives us which file to request. For example, this
can be "/index.html" or any complex path such as "/other/index.html".
We also took into consideration that if "/" was passed, we would take
on index.html by default. We then pre-pended the content directory
path into the request, making sure the web server wasn't allowed to 
grab anything else into the directory. Note by default ".." in request
parameters are disallowed in web browsers. Because of the strangeness
of web browsers using connection keep-alive, we had to implement
the response header of "Content-Length:" in bytes. At the same time
we detect the file size of the requested file, we also detect if
the file exists or not. If it does not, we stop everything we're doing
and just send a "HTTP/1.1 404 Not found" header as described in the
RFC 2616 and 4229 documentation for web servers. If the file does exist,
we then find the file size with strlen() and append that to the request
header of "Content-Length: 638", 638 being an example length.\\
\\
We then need to tell the client browser how to analyze the content
we need to give it. By RFC documentation, we need to send it the
response header of "Content-Type:". To do so in a simplistic manner,
we took the request file and processed the end extention and tried
matching it with a list of known types. In our case, the known types
we gave our server was HTML, Javascript, PNG, JPG and PHP. We then
had to translate these file types into the mime type is a file 
identifier docuemnted in the RFC 2046 documentation and can be
summarized at \url{http://en.wikipedia.org/wiki/Internet_media_type}.
An example mime type would be "text/html" or "application/javascript".
Once we've determined the content type, we append another response header
such as "Content-Type: application/javascript".\\
\\
By this point, we have completed the request header build for
our web server. We then send this to the browser to interpret, following
the actual content of the file along witha "HTTP/1.1 200 OK" header.
From there, we do not close the connection and let the browser handle
the closing of the connection due to the default of "keep-alive" with
the browsers. Now it is known that we should properly handle this
by interpreting the request header, but due to the time restrictions
this was not possible.\\
\\
\subsection{SSL Encryption Implementation}

In order to begin a SSL connection, we needed to generate self-signed
certificates for the web server to give and serve to the browser. To
do so, we followed instructions similar to this web page.
\url{http://www.akadia.com/services/ssh_test_certificate.html}.\\
\\
In the main function of our program, we first setup our TLS connection
and its default parameters. Its like setting up the socket types in 
web browsers. But in this case, we have to define the type of certificates
we're using, the type of connection it will be using and more. Now the server
requires a key and certificate. However, there are some intermediate 
certificates and so on that can be used (if there's a third party involved
in the verification). But in this case, we will not be using them.\\
\\
After we set the TLS stage, we then create a normal web socket,
using the default way of creating them. The only difference is when
we receive requests from the client browser. When the browser sends us
a request, we accept the request and make a file descriptor based on it.
We then set the TLS pointer to the file descriptor where it will check
if it requires an SSL connection. If the browser did not request a secure
connection, it will fail and stop there. If it does, then we perform the 
proper handshake by sending the certificate to the browser for verification.
When the verification is complete, a secure connection is established
between the client and the server.\\
\\
From here, we use our normal ways of passing and receiving data through
the file descriptors, except we call GNU TLS functions instead, which
will handle our decryption and encryption for us. This concludes
the SSL encryption of the web server where it'll terminate the GNU TLS
connection between the browser and server.\\
\\
\subsection{PHP Execution}

During the file type handling of the web server implementation, we
detect if the script is PHP. If the script is of mime type "application/php"
we close the current file descriptor which opened the file, and
create a different one in its place. In fact there is a nice function
in C called popen which will execute a command line in "/bin/sh" and
throw it output into a file descriptor. From there on, we change the 
mime type into "text/html" since the output should be of that type
and we parse the file descriptor normally. Now if there's a script error,
the web server by default will send the errors to the browser. To disable
this you would need to edit PHP's configuration file.\\
\\
\section{Conclusion and Remarks}

This entire application was built by myself, I hope you can understand
the time constraint of the project and how I implemented many different
parts of a full project that a team would build. I had a ton a fun
building this web server and I hope you don't penalize me for the fact
that I wanted to work by myself on this assignment.\\
\\
\section{Ammendments}

Because of the limitations of GNUTLS, unfortunately we had to chunk the
data if we were sending files larger than 16KB. To do so we chunk it up
and send bit by bit.

% include the bibliography but using bibtex
\bibliographystyle{plain}
\bibliography{bibfile1,bibfile2,bibfile3}
 
\end{document}
